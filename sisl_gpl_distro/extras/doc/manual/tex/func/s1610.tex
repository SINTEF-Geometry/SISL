\subsection{Draw ``polygons'' in a surface using a recursive algorithm testing for 
planarity of the subdivided parts.}
\funclabel{s1610}
\begin{minipg1}
To draw a surface using recursive subdivision.
This drawing function is slower than the one drawing constant parameter
lines, s1237() (page~\pageref{s1237}). But the recursive algorithm
will guarantee that all variations in the surface greater
than the surface resolution will be drawn.
\end{minipg1} \\ \\
SYNOPSIS\\
	\>void s1610(\begin{minipg3}
		{\fov surf}, {\fov epscur}, {\fov epssur}, {\fov stat})
		\end{minipg3}\\[0.3ex]
		\>\>	SISLSurf	\> 	*{\fov surf};\\
		\>\>	double	\>	{\fov epscur};\\
		\>\>	double	\>	{\fov epssur};\\
		\>\>	int 	\>	*{\fov stat};\\
\\
ARGUMENTS\\
	\>Input Arguments:\\
	\>\>	{\fov surf}\> - \>	\begin{minipg2}
				The NURBS surface to be drawn.
				\end{minipg2}\\
	\>\>	{\fov epscur}\> - \>	\begin{minipg2}
				Geometry resolution for curve.
				The maximal distance allowed between
				the curves 
				drawn and the surface.
				\end{minipg2}\\[0.8ex]
	\>\>	{\fov epssur}\> - \>	\begin{minipg2}
				Geometry resolution for surface.
				\end{minipg2}\\
\\
	\>Output Arguments:\\
	\>\>	{\fov stat}	\> - \>	Status messages\\
		\>\>\>\>\>		$> 0$	: warning\\
		\>\>\>\>\>		$= 0$	: ok\\
		\>\>\>\>\>		$< 0$	: error\\
\\
NOTE\\
\>\begin{minipg6}
This function calls s6drawseq() which
is device dependent. Before using the function make sure 
you have a version of s6drawseq() 
interfaced to your graphic package.
More about s6drawseq() on page~\pageref{s6drawseq}.
\end{minipg6}
\\ 
\newpagetabs
EXAMPLE OF USE\\
		\>	\{ \\
		\>\>	SISLSurf	\> 	*{\fov surf};\\
		\>\>	double	\>	{\fov epscur};\\
		\>\>	double	\>	{\fov epssur};\\
		\>\>	int 	\>	{\fov stat};\\
		\>\>	\ldots \\
	\>\>s1610(\begin{minipg4}
		{\fov surf}, {\fov epscur}, {\fov epssur}, \&{\fov stat});
			\end{minipg4}\\
		\>\>	\ldots \\
		\>	\}
\end{tabbing}
