\subsection{Approximate a conic arc with a curve.}
\funclabel{s1611}
\begin{minipg1}
  To approximate a conic arc with a curve in two or three
  dimensional space. If two points are given, a straight line is
  produced, if three an approximation of a circular arc, and if four or
  five a conic arc.
  The output will be represented as a B-spline curve.
\end{minipg1} \\ \\
SYNOPSIS\\
        \>void s1611(\begin{minipg3}
                {\fov point}, {\fov numpt}, {\fov dim}, {\fov typept}, {\fov open}, {\fov order},
                {\fov startpar}, {\fov epsge}, {\fov endpar}, {\fov curve}, {\fov stat})
                \end{minipg3}\\[0.3ex]
                \>\>    double  \>      {\fov point}[\,];\\
                \>\>    int     \>      {\fov numpt};\\
                \>\>    int     \>      {\fov dim};\\
                \>\>    double  \>      {\fov typept}[\,];\\
                \>\>    int     \>      {\fov open};\\
                \>\>    int     \>      {\fov order};\\
                \>\>    double  \>      {\fov startpar};\\
                \>\>    double  \>      {\fov epsge};\\
                \>\>    double  \>      *{\fov endpar};\\
                \>\>    SISLCurve       \>      **{\fov curve};\\
                \>\>    int     \>      *{\fov stat};\\
\\
ARGUMENTS\\
        \>Input Arguments:\\
        \>\>    {\fov point}    \> - \>
                \begin{minipg2}
                  Array of length $dim\times numpt$ containing the
                  points/ derivatives to be interpolated.
                \end{minipg2}\\[0.3ex]
        \>\>    {\fov numpt}    \> - \>
                                No. of points/derivatives in the
                                point array.
                                \\
        \>\>    {\fov dim}      \> - \> \begin{minipg2}
                                The dimension of the space in which
                                the points lie.
                                \end{minipg2}\\
        \>\>    {\fov typept}   \> - \> \begin{minipg2}
                                Array (length numpt) containing type
                                indicator for points/derivatives/
                                second-derivatives:
                                \end{minipg2} \\[0.3ex]
                \>\>\>\>\>      1 : Ordinary point.\\
                \>\>\>\>\>      3 : Derivative to next point.\\
                \>\>\>\>\>      4 : Derivative to prior point.\\
        \>\>    {\fov open}     \> - \> Open or closed curve:\\
                \>\>\>\>\>      0 :     Closed curve, not implemented.\\
                \>\>\>\>\>      1 :     Open curve.\\
        \>\>    {\fov order}    \> - \> \begin{minipg2}
                                The order of the B-spline curve
                                to be produced.
                                \end{minipg2}\\
        \>\>    {\fov startpar}\> - \>  \begin{minipg2}
                                Parameter-value to be used at the
                                start of the curve.
                                \end{minipg2}\\
        \>\>    {\fov epsge}    \> - \> The geometry resolution.\\
\\
\newpagetabs
        \>Output Arguments:\\
        \>\>    {\fov endpar}   \> - \> \begin{minipg2}
                                Parameter-value used at the end
                                of the curve.
                                \end{minipg2}\\
        \>\>    {\fov curve}    \> - \> Pointer to the output B-spline curve.\\
        \>\>    {\fov stat}     \> - \> Status messages\\
                \>\>\>\>\>              $> 0$   : warning\\
                \>\>\>\>\>              $= 0$   : ok\\
                \>\>\>\>\>              $< 0$   : error\\
\\
NOTE\\
\>\begin{minipg6}
When four points/tangents are given as input, the xy term of the
implicit equation is set to zero. Thus the points might end on two
branches of a hyperbola and a straight line is produced. When
four or five points/tangents are given only three of these should
actually be points.
\end{minipg6}
\\ \\
EXAMPLE OF USE\\
                \>      \{ \\
                \>\>    double  \>      {\fov point}[30];\\
                \>\>    int     \>      {\fov numpt} = 10;\\
                \>\>    int     \>      {\fov dim} = 3;\\
                \>\>    double  \>      {\fov typept}[10];\\
                \>\>    int     \>      {\fov open};\\
                \>\>    int     \>      {\fov order};\\
                \>\>    double  \>      {\fov startpar};\\
                \>\>    double  \>      {\fov epsge};\\
                \>\>    double  \>      {\fov endpar};\\
                \>\>    SISLCurve       \>      *{\fov curve};\\
                \>\>    int     \>      {\fov stat};\\
                \>\>    \ldots \\
        \>\>s1611(\begin{minipg4}
                {\fov point}, {\fov numpt}, {\fov dim}, {\fov typept}, {\fov open}, {\fov order}, {\fov startpar}, {\fov epsge},\linebreak
                \&{\fov endpar}, \&{\fov curve}, \&{\fov stat});
                        \end{minipg4}\\
                \>\>    \ldots \\
                \>      \} \\
\end{tabbing}
