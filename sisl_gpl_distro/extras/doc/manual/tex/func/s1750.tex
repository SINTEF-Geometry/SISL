\subsection{Express a curve using a higher order basis.}
\funclabel{s1750}
\begin{minipg1}
  To describe a curve using a higher order basis.
\end{minipg1} \\ \\
SYNOPSIS\\
        \>void s1750(\begin{minipg3}
        {\fov curve}, {\fov order}, {\fov newcurve}, {\fov stat})
                \end{minipg3}\\[0.3ex]
                \>\>    SISLCurve       \>      *{\fov curve};\\
                \>\>    int     \>      {\fov order};\\
                \>\>    SISLCurve       \>      **{\fov newcurve};\\
                \>\>    int     \>      *{\fov stat};\\
\\
ARGUMENTS\\
        \>Input Arguments:\\
        \>\>    {\fov curve}    \> - \> The input curve.\\
        \>\>    {\fov order}    \> - \> \begin{minipg2}
                                Order of the new curve.
                                \end{minipg2}\\
\\
        \>Output Arguments:\\
        \>\>    {\fov newcurve}\> - \>\begin{minipg2}
                                The new curve of higher order.
                                \end{minipg2}\\
        \>\>    {\fov stat}     \> - \> Status messages\\
                \>\>\>\>\>              $> 0$   : warning\\
                \>\>\>\>\>              $= 0$   : ok\\
                \>\>\>\>\>              $< 0$   : error\\
\\
EXAMPLE OF USE\\
                \>      \{ \\
                \>\>    SISLCurve       \>      *{\fov curve};\\
                \>\>    double  \>      {\fov order};\\
                \>\>    SISLCurve       \>      *{\fov newcurve};\\
                \>\>    int     \>      stat;\\
                \>\>    \ldots \\
        \>\>s1750(\begin{minipg4}
                {\fov curve}, {\fov order}, \&{\fov newcurve}, \&{\fov stat});
                        \end{minipg4}\\
                \>\>    \ldots \\
                \>      \}
\end{tabbing}
