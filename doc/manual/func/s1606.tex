\subsection{Compute a blending curve between two curves.}
\funclabel{s1606}
\begin{minipg1}
  To compute a blending curve between two curves.
  Two points indicate between which
  ends the blend is to be produced.
  The blending curve is either a circle or
  an approximated conic section if this is
  possible, otherwise it is a quadratic polynomial spline curve.
  The output is represented as a B-spline curve.
\end{minipg1} \\ \\
SYNOPSIS\\
        \>void s1606(\begin{minipg3}
        {\fov curve1}, {\fov curve2}, {\fov epsge}, {\fov point1}, {\fov point2},
        {\fov blendtype}, {\fov dim}, {\fov order}, {\fov newcurve}, {\fov stat})
                \end{minipg3}\\[0.3ex]
                \>\>    SISLCurve       \>      *{\fov curve1};\\
                \>\>    SISLCurve       \>      *{\fov curve2};\\
                \>\>    double  \>      {\fov epsge};\\
                \>\>    double  \>      {\fov point1}[\,];\\
                \>\>    double  \>      {\fov point2}[\,];\\
                \>\>    int     \>      {\fov blendtype};\\
                \>\>    int     \>      {\fov dim;}\\
                \>\>    int     \>      {\fov order};\\
                \>\>    SISLCurve       \>      **{\fov newcurve};\\
                \>\>    int     \>      *{\fov stat};\\
\\
ARGUMENTS\\
        \>Input Arguments:\\
        \>\>    {\fov curve1}   \> - \> The first input curve.\\
        \>\>    {\fov curve2}   \> - \> The second input curve.\\
        \>\>    {\fov epsge}    \> - \> Geometry resolution.\\
        \>\>    {\fov point1}   \> - \> \begin{minipg2}
                        Point near the end of curve 1 where the blend starts.
                                \end{minipg2}\\
        \>\>    {\fov point2}   \> - \> \begin{minipg2}
                        Point near the end of curve 2 where the blend starts.
                                \end{minipg2}\\
        \>\>    {\fov blendtype}\> - \> Indicator of type of blending.\\
                \>\>\>\>\>      $=1$ : \>\begin{minipg5}
                                Circle, interpolating tangent on first
                                curve, not on curve 2, if possible.
                                \end{minipg5}\\[0.3ex]
                \>\>\>\>\>      $=2$ : \>\begin{minipg5}
                                Conic if possible,
                                \end{minipg5}\\
                \>\>\>\>\>      else : \>\begin{minipg5}
                                polynomial segment.
                                \end{minipg5}\\
        \>\>    {\fov dim}      \> - \> Dimension of the geometry space.\\
        \>\>    {\fov order}    \> - \> Order of the blending curve.\\
\\
        \>Output Arguments:\\
        \>\>    {\fov newcurve}\> - \> Pointer to the B-spline blending curve.\\
        \>\>    {\fov stat}     \> - \> Status messages\\
                \>\>\>\>\>              $> 0$   : warning\\
                \>\>\>\>\>              $= 0$   : ok\\
                \>\>\>\>\>              $< 0$   : error\\
\newpagetabs
EXAMPLE OF USE\\
                \>      \{ \\
                \>\>    SISLCurve       \>      *{\fov curve1};\\
                \>\>    SISLCurve       \>      *{\fov curve2};\\
                \>\>    double  \>      {\fov epsge};\\
                \>\>    double  \>      {\fov point1}[3];\\
                \>\>    double  \>      {\fov point2}[3];\\
                \>\>    int     \>      {\fov blendtype};\\
                \>\>    int     \>      {\fov dim} = 3;\\
                \>\>    int     \>      {\fov order};\\
                \>\>    SISLCurve       \>      *{\fov newcurve};\\
                \>\>    int     \>      {\fov stat};\\
                \>\>    \ldots \\
        \>\>s1606(\begin{minipg4}
        {\fov curve1}, {\fov curve2}, {\fov epsge}, {\fov point1}, {\fov point2},
        {\fov blendtype}, {\fov dim}, {\fov order}, \&{\fov newcurve}, \&{\fov stat});
                        \end{minipg4}\\
                \>\>    \ldots \\
                \>      \} \\
\end{tabbing}
