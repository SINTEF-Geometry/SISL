\subsection{Calculate area, moment of inertia and weightpoint of area limited by 2-D curve.}
\funclabel{s1242}
\begin{minipg1}
Compute the area surrounded by a 2D-curve and the straight lines from a reference
point to the end points of the curve. The area might have a negative value
dependent on the direction of the curve relative to the reference point. When
used on a closed curve, the routine gives as output the area surrounded by the curve.
The difference between the area calculated and the real area will be less than
(aepsco/the area calculated). The weight point of the surface and the rotational
momentum in accordance with a given axis are also computed.
\end{minipg1} \\ \\
SYNOPSIS\\
        \>void s1242(\begin{minipg3}
        {\fov curve}, {\fov point}, {\fov dim}, {\fov epsco}, {\fov weight}, {\fov area}, {\fov moment}, {\fov stat})
                \end{minipg3}\\[0.3ex]
                \>\>    SISLCurve       \>      *{\fov curve};\\
                \>\>    double  \>      {\fov point}[\,];\\
                \>\>    int     \>      {\fov dim};\\
                \>\>    double  \>      {\fov epsco};\\
                \>\>    double  \>      {\fov weight}[\,];\\
                \>\>    double  \>      *{\fov area};\\
                \>\>    double  \>      *{\fov moment};\\
                \>\>    int     \>      *{\fov stat};\\
\\
ARGUMENTS\\
        \>Input Arguments:\\
        \>\>    {\fov curve}    \> - \> B-spline curve.\\
        \>\>    {\fov point}    \> - \>\begin{minipg2}
                                Reference point in the area computation / axis vertical
                                to a plane (given as point in 2D).
                                \end{minipg2}\\
        \>\>    {\fov dim}\> - \>\begin{minipg2}
                                Dimension of the space in which epoint lies. The
                                dimension is supposed to be equal to 2.
                                \end{minipg2}\\[0.3ex]
        \>\>    {\fov epsco} \> - \>\begin{minipg2}
                                Resolution.
                                \end{minipg2}\\
\\
        \>Output Arguments:\\
        \>\>    {\fov weight}   \> - \>\begin{minipg2}
                                Weight point.
                                \end{minipg2}\\
        \>\>    {\fov area}     \> - \>\begin{minipg2}
                                The area calculated.
                                \end{minipg2}\\
        \>\>    {\fov moment}   \> - \>\begin{minipg2}
                                Rotational momentum.
                                \end{minipg2}\\
        \>\>    {\fov stat}     \> - \> Status messages\\
                \>\>\>\>\>              $> 0$   : warning\\
                \>\>\>\>\>              $= 0$   : ok\\
                \>\>\>\>\>              $< 0$   : error\\
\\
NOTE\\
\>\begin{minipg6}
        The algorithm is based on recursive
        subdivision and will thus for small values
        of epsco require long computation time.
\end{minipg6} \\ \\
EXAMPLE OF USE\\
                \>      \{ \\
                \>\>    SISLCurve       \>      *{\fov curve};\\
                \>\>    double  \>      {\fov point[2]};\\
                \>\>    int     \>      {\fov dim};\\
                \>\>    double  \>      {\fov epsco};\\
                \>\>    double  \>      {\fov weight[2]};\\
                \>\>    double  \>      {\fov area};\\
                \>\>    double  \>      {\fov moment};\\
                \>\>    int     \>      {\fov stat};\\
                \>\>    \ldots \\
        \>\>s1242(\begin{minipg4}
                {\fov curve}, {\fov point}, {\fov dim}, {\fov epsco}, {\fov weight}, \&{\fov area}, \&{\fov moment}, \&{\fov stat});
                        \end{minipg4}\\
                \>\>    \ldots \\
                \>      \}
\end{tabbing}
