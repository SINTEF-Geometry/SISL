\subsection{Compute a rotational swept surface.}
\funclabel{s1302}
\begin{minipg1}
  To create a rotational swept surface by rotating a curve
  a given angle around the axis defined by {\fov point}[\,] and
  {\fov axis}[\,].
  The maximal deviation allowed between the true rotational surface and the
  generated surface, is {\fov epsge}.
  If {\fov epsge} is set to 0, a NURBS surface is generated and if
  $epsge >0$, a B-spline surface is generated.
\end{minipg1} \\ \\
SYNOPSIS\\
        \>void s1302(\begin{minipg3}
                        {\fov curve}, {\fov epsge}, {\fov angle}, {\fov point}, {\fov axis}, {\fov surf}, {\fov stat})
                \end{minipg3}\\[0.3ex]
                \>\>    SISLCurve       \>      *{\fov curve};\\
                \>\>    double  \>      {\fov epsge};\\
                \>\>    double  \>      {\fov angle};\\
                \>\>    double  \>      {\fov point}[\,];\\
                \>\>    double  \>      {\fov axis}[\,];\\
                \>\>    SISLSurf        \>      **{\fov surf};\\
                \>\>    int     \>      *{\fov stat};\\
\\
ARGUMENTS\\
        \>Input Arguments:\\
        \>\>    {\fov curve}    \> - \> \begin{minipg2}
                                Pointer to the curve that is to be rotated.
                                \end{minipg2}\\
        \>\>    {\fov epsge}    \> - \> \begin{minipg2}
                                Maximal deviation allowed between the true rotational
                                surface and the generated surface.
                                \end{minipg2}\\[0.3ex]
        \>\>    {\fov angle}    \> - \> \begin{minipg2}
                        The rotational angle. The angle is counterclockwise around axis. If the absolute
                        value of the angle is greater than $2\pi$ then a rotational surface that is
                        closed in the rotation direction is made.
                                \end{minipg2}\\[0.8ex]
        \>\>    {\fov point}    \> - \> \begin{minipg2}
                                Point on the rotational axis.
                                \end{minipg2} \\
        \>\>    {\fov axis}     \> - \> \begin{minipg2}
                                Direction of rotational axis.
                                \end{minipg2} \\
\\
        \>Output Arguments:\\
        \>\>    {\fov surf}     \> - \> \begin{minipg2}
                                        Pointer to the produced surface.
                                        This will be a NURBS (i.e.\
                                        rational) surface if $epsge=0$
                                        and a \mbox{B-spline} (i.e.\
                                        non-rational) surface if $epsge>0$.
                                \end{minipg2}\\
        \>\>    {\fov stat}     \> - \> Status messages\\
                \>\>\>\>\>              $>0$    : warning\\
                \>\>\>\>\>              $=0$    : ok\\
                \>\>\>\>\>              $<0$    : error\\
\newpagetabs
EXAMPLE OF USE\\
                \>      \{ \\
                \>\>    SISLCurve       \>      *{\fov curve};\\
                \>\>    double  \>      {\fov epsge};\\
                \>\>    double  \>      {\fov angle};\\
                \>\>    double  \>      {\fov point}[3];\\
                \>\>    double  \>      {\fov axis}[3];\\
                \>\>    SISLSurf        \>      *{\fov surf};\\
                \>\>    int     \>      {\fov stat};\\
                \>\>    \ldots \\
        \>\>    s1302(\begin{minipg4}
                        {\fov curve}, {\fov epsge}, {\fov angle}, {\fov point}, {\fov axis}, \&{\fov surf}, \&{\fov stat});
                        \end{minipg4}\\
                \>\>    \ldots\\
                \>      \}
\end{tabbing}
